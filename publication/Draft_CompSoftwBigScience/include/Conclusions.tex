A novel system for the dynamic, on-demand provisioning of virtual machines to run jobs
in a high energy physics context on an external, not dedicated resource as
realized at the HPC cluster \NEMO at the University of Freiburg has been
implemented. An interface between the schedulers of the
host system and the external system from which requests are sent is needed to
monitor and steer jobs in a scalable way. For this workflow the cloud meta-scheduler \Roced
has been implemented and deployed for the described use-cases.
The approach can be adapted to work with other platforms and could be extended to
container technologies like Singularity~\cite{VRE2017}.

The CPU performance and usage of the setup have been analyzed for the job execution environment.
The expected performance loss due to the virtualization has been found to be
sufficiently small to be compensated by the added flexibility and other benefits
of this setup.

A possible extension of such a virtualized setup is the provisioning of functionalities
for snapshots and migration of jobs. This would facilitate the efficient integration of
long-running monolithic jobs into HPC clusters.

The provided solution extends the available compute ressources for HEP calculations
and could be one possibilitly to cope with new data from the upcoming High-Luminosity
upgrade of the LHC. Since HEP VREs are perfect for backfilling this could be used on various
cluster ressources.
