The open-source project \HTCondor provides a workload management system which is highly configurable and modular~\cite{HTCondor}. Batch processing workflows can be submitted and are then forwarded by \HTCondor to idle resources. \HTCondor maintains a resource pool, which worker nodes in a local or remote cluster can join. Once \HTCondor has verified the authenticity and features of the newly joined machines, computing jobs are automatically transferred. Special features are available to connect from within isolated network zones, e.g. via a Network Address Translation Portal, to the central \HTCondor pool. The Connection Brokering (CCB) service~\cite{HTCondorCCB} is especially valuable to connect virtual machines to the central pool. These features and the well-known ability of \HTCondor to scale to O(100k) of parallel batch jobs makes \HTCondor well suited as a workload management system for the use cases described in this paper.

The virtual machines spawned for the CMS user group of the KIT come with \texttt{startd} the \HTCondor client pre-installed. % SATZ hoert sich komisch an, Bitte reparieren
This client is started after the machine has fully booted and connects to the central \HTCondor pool at the KIT via a shared secret. Due to \HTCondor's dynamic design, new machines in the pool will automatically receive jobs and the transfer of the job configuration and meta-data files is handled via \HTCondor's internal file transfer systems.
