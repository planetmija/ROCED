Hardware virtualization has become mainstream technology over the last decade as it allows
both to host more than one operating system on a single server and to strictly
separate users of software environments.
Hardware and software stacks are decoupled and therefore complete software
environment can be migrated across hardware boundaries.
While widespread in computer center
operation this technique is rarely applied in HPC.
% it is still skeptically seen in the field of scientific computing~\cite{VirtualisationScientificComp} and
% thus rarely applied in HPC.

\subsection{Computing at the University of Freiburg}
% Infrastructure setting

The computer center at the University of Freiburg provides
medium scaled research
infrastructures like cloud, storage, and especially HPC services adapted to the
needs of various scientific communities. Significant standardization
in hardware and software is necessary for the operation of compute systems comprised of
more than 1000 individual nodes combined with a small group of administrators.

The level of granularity of the software stack provided is not fine enough to
directly support the requirements of world-wide efforts like the
ATLAS or CMS experiments.
Therefore, novel approaches are necessary to ensure optimal use of the system and to open the cluster to as many different use-cases as
possible without increasing the operational effort.
Transferring expertise from the operation of the established local
private cloud, %in cooperation with the ViCE project~\cite{vice},
the use of \Openstack as a cloud platform has been identified
as a
suitable solution for \NEMO. This approach provides a user defined software
deployment in addition to the existing software module system.
The resulting challenges range from the automated creation of suitable
virtual machines to their on-demand deployment and scheduling.

\subsection{Research Cluster NEMO}

The research cluster \NEMO is a cluster for state-wide
research in the scientific fields Elementary Particle Physics, Neuroscience and
Microsystems Engineering. It started its operation on the 1st of August 2016
%with then 748 nodes having 20 physical cores and 128\,GiB of RAM each.
and consists currently of 900 nodes with 20 physical cores and 128\,GiB of RAM each.
Omni-Path~\cite{Omnipath} spans a high speed low latency network of 100\,Gbit/s between nodes.
The parallel storage is based on \BeeGFS~\cite{BeeGFS} with
% 576\,TB capacity.
768\,TB capacity.
%In October 2017 the \NEMO cluster was extended by research groups and offers
%currently 900 nodes and a total capacity of 768\,TB of parallel storage.

\subsection{Separation of software environments}

The file system of a virtual machine or VRE is a
disk image presented as a single file. From the computer center's perspective
this image is a ``black box'' requiring no involvement or efforts like
updates of the operating system or the provisioning of software packages of a
certain version. From the researcher's perspective the VRE is an individual
virtual node whose operating system, applications and configurations
as well as certain hardware-level parameters, e.g. CPU and RAM, can be
configured fully autonomously by the researcher within agreed upon limits.


To increase the flexibility in hosted software environments, the standard bare metal
operation of \NEMO is extended with an installation of \Openstack
components~\cite{hpc-symp:2016}.
The \NEMO cluster uses Adaptive's Workload Manager \Moab~\cite{Moab} as a
scheduler of compute jobs.
\Openstack as well can schedule virtual machines on the same nodes and
resources.
To avoid conflicts, it is necessary to define the master scheduler
which decides the job assignment to the worker nodes.
Both \Moab and \Openstack are
unaware that another scheduler exists within the cluster and there is
no API which enables them to  communicate with each other. Since the majority of users still use the
bare metal HPC cluster, \Moab is deployed as the primary scheduler. It allows for
detailed job description and offers sophisticated scheduling features like
fair-share, priority-based scheduling, detailed limits,
etc. \Openstack 's task is to deploy the virtual machines, but \Moab will initially start the VRE
jobs and the VRE job will instruct \Openstack to start the virtual machine on the
reserved resources with the required flavor, i.e. the resource definition in \Openstack.

When a VRE job is submitted to the \NEMO cluster, \Moab will first calculate the
priority and the needed resources of the job and then inserts it into its queue.
When the job is in line for execution and the requested resources are available,
the job will start a script which then starts the VRE on the selected node
within the resource boundaries. During the run-time of the VRE a monitoring
script regularly checks if the VRE is still running and terminates the job when
the VRE has ended.
When the job ends, \Openstack gets a signal to terminate the virtual machine and
the VRE job ends as well.  Neither \Moab nor \Openstack have access
inside the VRE, so they cannot assess if the VRE is actually busy or
idle.
The software package \Roced (described in
further detail in Section~\ref{section:roced}) has been introduced to
solve this issue.
It is used as a broker between
different HPC schedulers,  translating resources and monitoring usage inside the
virtual machine, as well as starting and stopping VRE images on demand.
